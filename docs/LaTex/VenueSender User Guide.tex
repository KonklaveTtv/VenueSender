\documentclass{article}
\usepackage[utf8]{inputenc}
\usepackage{fontspec}
\usepackage{xcolor}
\usepackage{geometry}
\usepackage{listings}
\usepackage{mdframed}

\setmainfont{Arial}

\definecolor{background}{RGB}{40,40,40}
\definecolor{textcolor}{RGB}{230,230,230}
\definecolor{accent}{RGB}{90,90,255}
\pagecolor{background}
\color{textcolor}

\geometry{a4paper, margin=0.8in, top=1.2in, bottom=1.2in}

\lstdefinestyle{mystyle}{
	backgroundcolor=\color{background},
	commentstyle=\color{accent},
	keywordstyle=\color{textcolor},
	numberstyle=\tiny\color{textcolor},
	stringstyle=\color{accent},
	basicstyle=\ttfamily\footnotesize\color{textcolor},
	breakatwhitespace=false,
	breaklines=true,
	captionpos=b,
	keepspaces=true,
	numbers=left,
	numbersep=5pt,
	showspaces=false,
	showstringspaces=false,
	showtabs=false,
	tabsize=2
}
\lstset{style=mystyle}

% Hyperref for links
\usepackage[colorlinks=true, allcolors=accent]{hyperref}

% Header/Footer setup for dark theme
\usepackage{fancyhdr}
\pagestyle{fancy}
\fancyhf{}
\lhead{\color{textcolor}VenueSender Codebase Overview}
\rhead{\color{textcolor}\thepage}
\renewcommand{\headrulewidth}{0.4pt}
\renewcommand{\footrulewidth}{0.4pt}
\renewcommand{\headrule}{\hbox to\headwidth{\color{accent}\leaders\hrule height \headrulewidth\hfill}}
\renewcommand{\footrule}{\hbox to\headwidth{\color{accent}\leaders\hrule height \footrulewidth\hfill}}

% Other packages
\usepackage{enumitem}
\usepackage{microtype}

\title{\color{textcolor}User Guide for VenueSender}
\begin{document}

\author{Created by Spencer Lievens}
\date{\today}

	\maketitle
	
	\section*{Introduction}
	Welcome to the VenueSender user guide! This tool facilitates managing and sending emails to a list of venues, ensuring efficient and streamlined communications. This guide provides step-by-step instructions to navigate and utilize the program effectively.
	
	\section*{Starting the Program}
	Execute the VenueSender executable. Upon loading, the program initializes configurations and data. Should any setup issues emerge, descriptive error messages guide you through resolution.
	
	\section*{Configuration Files}
	The program relies on specific configuration files to operate correctly:
	\begin{itemize}
		\item \texttt{venues.csv}: Contains venue data. Ensure it is located in the root directory.
		\item \texttt{config.json}: Holds various configurations like SMTP settings and email details. It should also be in the root directory.
		\item For testing purposes, \texttt{mock\_venues.csv} and \texttt{mock\_config.json} should be located in the \texttt{src/test/} directory.
	\end{itemize}
	
	\section*{Venue Data Format}
	The \texttt{venues.csv} file should be structured with six columns: name, email, genre, state, city, and capacity. Ensure this format is maintained to prevent reading errors.
	
	\section*{Main Menu}
	After initiation, the main menu presents several options:
	\begin{itemize}
		\item Filter by Genre: Select venues based on genre.
		\item Filter by State: Choose venues from a particular state.
		\item Filter by City: Filter venues by city.
		\item Filter by Capacity: Opt for venues based on capacity.
		\item Clear Selected Venues: Reset your venue selections.
		\item View Selected Venues: Review your chosen venues.
		\item Show Email Settings: View email configurations.
		\item View \& Edit Email: Inspect and modify the current email draft.
		\item Email Custom Address: Dispatch an email to a custom address.
		\item Finish \& Send Emails: Dispatch emails to the selected venues.
		\item Exit VenueSender: Shut down the program.
	\end{itemize}
	
	\section*{Filtering Venues}
	VenueSender offers you the flexibility to tailor your audience by filtering venues based on various criteria.
	
	\subsection*{Filter by Genre}
	VenueSender can identify unique genres from the list of venues. To filter:
	\begin{itemize}
		\item Available Genres: The program displays a list of genres.
		\item Selection: Input comma-separated indices corresponding to desired genres.
	\end{itemize}
	
	\subsection*{Filter by State}
	Choose venues from specific states:
	\begin{itemize}
		\item Available States: A list of states is presented.
		\item Selection: Enter comma-separated indices of your preferred states.
	\end{itemize}
	
	\subsection*{Filter by City}
	\begin{itemize}
		\item Available Cities: A list of cities will be displayed.
		\item Selection: Specify comma-separated indices of cities you're targeting.
	\end{itemize}
	
	\subsection*{Filter by Capacity}
	For targeting venues based on seating:
	\begin{itemize}
		\item Available Capacities: VenueSender showcases various venue capacities.
		\item Selection: Indicate comma-separated indices that match the capacities you're interested in.
	\end{itemize}
	
	\section*{Reviewing and Resetting Selections}
	After filtering:
	\begin{itemize}
		\item View Selected Venues: Check the venues you've picked.
		\item Clear Selected Venues: Use this to reconsider or correct any mistaken selections.
	\end{itemize}
	
	\section*{Managing Your Email}
	\subsection*{Constructing the Email}
	Upon selecting View \& Edit Email:
	\begin{itemize}
		\item Subject: Enter the email subject. Remember, it shouldn't exceed a specified character limit. If it does, an error message will prompt you to shorten it.
		\item Message: Write the email content. Be wary of the character limit. If exceeded, you'll be alerted to condense your message.
		\item Attachment: Attach files by providing their paths. Ensure the attachment size doesn't surpass the limit (e.g., 24MB). If the file is too large, you'll be notified to pick a smaller file.
	\end{itemize}
	
	\subsection*{Email Password Encryption}
	For added security, the program encrypts the email password stored in the configuration file. If the password is not encrypted, the program will automatically encrypt it upon initialization. 
	
	\subsection*{Email Settings}
	Access Show Email Settings to review:
	\begin{itemize}
		\item SMTP Server
		\item SMTP Port
		\item Sender Email
		\item SSL Status
		\item verifyPeer Status
		\item verifyHost Status
		\item Verbose Mode
	\end{itemize}

	\subsection*{Sending Emails}
	Finish \& Send Emails initiates the process:
	\begin{itemize}
		\item The program checks the sender's email format. If it's incorrect, an error message instructs you to adjust it in your custom.json file.
		\item The program then checks if more than 49 venues have been selected, if so it sends using "Bcc", if less than 49 venues are selected then it sends an individual email to each.
		\item Connects to the SMTP server and authenticates.
		\item Emails are dispatched either individually or if more than 49 venues have been selected it sends via "Bcc:". For each, you'll receive a status update. Should any sending issues arise, you'll be informed (e.g., connection errors, authentication failures).
		\item Once all emails are sent, a completion message appears. Press return to continue.
	\end{itemize}
	
	\subsection*{Custom Address Email Sending}
	Access Show Email Settings to review:
	\begin{itemize}
		\item The user sets the custom email adress to be used.
		\item The program checks the sender's email format. If it's incorrect, an error message instructs you to adjust it in your custom.json file.
		\item Connects to the SMTP server and authenticates.
		\item The email is dispatched and you receive a status update. Should any sending issues arise, you'll be informed (e.g., connection errors, authentication failures).
		\item Once the email is sent, a completion message appears. Press return to continue.
	\end{itemize}

	\section*{Exiting the Program}
	To leave:
	\begin{itemize}
		\item Exit VenueSender: Choose this from the main menu. A confirmation ensures your intent to close the program.
	\end{itemize}
	
	\section*{Troubleshooting \& Error Handling}
	VenueSender is engineered to offer informative error messages:
	\begin{itemize}
		\item If a problem arises, read the displayed message. It typically instructs on rectification steps.
		\item Restarting the program might help.
		\item For persistent issues, consult the FAQ or contact the support team.
	\end{itemize}
	
	\section*{Understanding Error Messages}
	VenueSender has built-in error handling to assist users in diagnosing problems. Here are some common errors and their solutions:
	\begin{itemize}
		\item \textbf{CONFIG\_OPEN\_ERROR}: The program can't access the configuration file. Ensure the file paths are correct and that the files exist.
		\item \textbf{INVALID\_DATA\_IN\_CSV}: The CSV file's structure is not as expected. Verify that it has the correct columns and format.
		\item \textbf{INVALID\_CAPACITY\_IN\_CSV}: The capacity value in the CSV file is not a valid number. Ensure all capacities are numeric.
		\item \textbf{CONFIG\_LOAD\_ERROR}: There was an issue loading configurations from the JSON file. Double-check the file's structure and content.
		% Add other relevant errors here
	\end{itemize}
	
	\section*{Conclusion}
	Thanks for choosing VenueSender! We hope this user guide aids in your software navigation. For further assistance or feedback, don't hesitate to reach out. Safe sending!
	
\end{document}
